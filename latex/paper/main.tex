%%%%%%%%%%%%%%%%%%%%%%%%%%%%%%%%%%%%%%%%%
% University Assignment Title Page 
% LaTeX Template
% Version 1.0 (27/12/12)
%
% This template has been downloaded from:
% http://www.LaTeXTemplates.com
%
% Original author:
% WikiBooks (http://en.wikibooks.org/wiki/LaTeX/Title_Creation)
%
% License:
% CC BY-NC-SA 3.0 (http://creativecommons.org/licenses/by-nc-sa/3.0/)
% 
% Instructions for using this template:
% This title page is capable of being compiled as is. This is not useful for 
% including it in another document. To do this, you have two options: 
%
% 1) Copy/paste everything between \begin{document} and \end{document} 
% starting at \begin{titlepage} and paste this into another LaTeX file where you 
% want your title page.
% OR
% 2) Remove everything outside the \begin{titlepage} and \end{titlepage} and 
% move this file to the same directory as the LaTeX file you wish to add it to. 
% Then add \input{./title_page_1.tex} to your LaTeX file where you want your
% title page.
%
%%%%%%%%%%%%%%%%%%%%%%%%%%%%%%%%%%%%%%%%%
%\title{Title page with logo}
%----------------------------------------------------------------------------------------
%	PACKAGES AND OTHER DOCUMENT CONFIGURATIONS
%----------------------------------------------------------------------------------------

\documentclass[11pt]{article}
\usepackage[italian]{babel}
\usepackage[utf8x]{inputenc}
\usepackage{amsmath}
\usepackage{graphicx}
\usepackage[colorinlistoftodos]{todonotes}
\usepackage{enumitem}
\usepackage{listings}
\usepackage{filecontents}
\usepackage{verbatim}
\usepackage{eurosym}
\usepackage[export]{adjustbox}

\usepackage[margin=3cm]{geometry}
\usepackage{listings}
\usepackage{xcolor}

\definecolor{codegreen}{rgb}{0,0.6,0}
\definecolor{codegray}{rgb}{0.5,0.5,0.5}
\definecolor{codepurple}{rgb}{0.58,0,0.82}
\definecolor{backcolour}{rgb}{0.95,0.95,0.92}

\lstdefinestyle{mystyle}{
    backgroundcolor=\color{backcolour},   
    commentstyle=\color{codegreen},
    keywordstyle=\color{magenta},
    numberstyle=\tiny\color{codegray},
    stringstyle=\color{codepurple},
    basicstyle=\ttfamily\footnotesize,
    breakatwhitespace=false,         
    breaklines=true,                 
    captionpos=b,                    
    keepspaces=true,                 
    numbers=left,                    
    numbersep=5pt,                  
    showspaces=false,                
    showstringspaces=false,
    showtabs=false,                  
    tabsize=2
}

\lstset{style=mystyle}

\begin{document}
\begin{titlepage}

\newcommand{\HRule}{\rule{\linewidth}{0.5mm}} % Defines a new command for the horizontal lines, change thickness here

\center % Center everything on the page
 
%----------------------------------------------------------------------------------------
%	HEADING SECTIONS
%----------------------------------------------------------------------------------------

\textsc{\LARGE Università di Messina}\\[1.5cm] % Name of your university/college
\textsc{\Large Dipartimento di scienze matematiche e informatiche, scienze fisiche e della terra}\\[0.5cm] % Major heading such as course name
\textsc{\large Corso di Laurea Triennale in Informatica}\\[0.5cm] % Minor heading such as course title

%----------------------------------------------------------------------------------------
%	TITLE SECTION
%----------------------------------------------------------------------------------------

\HRule \\[0.4cm]
{ \huge \bfseries Distributed computation of linear algebra operations}\\[0.4cm] % Title of your document
\HRule \\[1.5cm]
 
%----------------------------------------------------------------------------------------
%	AUTHOR SECTION
%----------------------------------------------------------------------------------------

\begin{minipage}{0.4\textwidth}
\begin{flushleft} \large
\emph{Author:}\\
Gabriele \textsc{Aloisio} \textit{(503264)} \\
\end{flushleft}
\end{minipage}
~
\begin{minipage}{0.4\textwidth}
\begin{flushright} \large
\emph{Supervisor:} \\
prof. Roberto \textsc{Marino} \\
\end{flushright}
\end{minipage}\\[2cm]

% If you don't want a supervisor, uncomment the two lines below and remove the section above
%\Large \emph{Author:}\\
%John \textsc{Smith}\\[3cm] % Your name

%----------------------------------------------------------------------------------------
%	DATE SECTION
%----------------------------------------------------------------------------------------

{\large \today}\\[2cm] % Date, change the \today to a set date if you want to be precise

%----------------------------------------------------------------------------------------
%	LOGO SECTION
%----------------------------------------------------------------------------------------

\includegraphics[width=70px, keepaspectratio]{unime.png}\\[1cm] % Include a department/university logo - this will require the graphicx package
 
%----------------------------------------------------------------------------------------

\vfill % Fill the rest of the page with whitespace

\end{titlepage}

\tableofcontents
\pagebreak

\section{Abstract}

\section{Introduction}
Matrix operations are fundamental in numerous scientific and computational domains, serving as the building blocks for various applications. However, traditional sequential computation on a single machine often becomes a bottleneck, limiting the speed and scalability of matrix calculations. To address this problem, we are going to use an open-source distributed computing framework called \textbf{Ray}.

\subsection{Overview of distributed systems and networks}
A distributed system consists of multiple interconnected computers that collaborate and coordinate their activities to achieve a common goal. These systems are designed to tackle tasks that cannot be efficiently computed by a single machine. Networks serve as the backbone of distributed systems, enabling communication among the connected nodes. Network infrastructures enable coordination, data sharing, and synchronization across distributed systems, regardless of their physical locations.

\pagebreak
\subsection{The Ray library}
The Ray library is an open-source distributed computing framework primarily designed for building scalable and high-performance applications. It was developed by the company Anyscale, which aimed to simplify the development of distributed and parallel computing applications. When it comes to matrix operation computation, using Ray for distributed computation offers several advantages over performing the operations on a single machine:
    \begin{itemize}
        \item \textbf{Faster execution}: With Ray you can use multiple machines and CPUs to compute operations. This significantly reduces the computation time compared to a single machine. Each machine can work on a subset of the matrix data, completing the calculations in parallel. This distributed approach can lead to substantial speedups.
        \item \textbf{Scalability}: As the size of the matrices grows, a single machine may struggle to handle the computational demands due to memory limitations or processing power constraints. Ray allows you to scale horizontally by adding more machines to the distributed setup.
        \item \textbf{Fault tolerance}: Ray offers fault tolerance mechanisms that ensure the continuity of computation even in the presence of failures. If a machine participating in the distributed computation fails, Ray can automatically redistribute the workload to other available machines.
        \item \textbf{Resource utilization}: With Ray, each machine contributes its processing power and memory capacity to the overall computation. This efficient utilization of resources allows you to make the most of the available hardware infrastructure, compared to a single machine that may be underutilized.
    \end{itemize}


\pagebreak
\section{Case study}

\section{Implementation}
    \subsection{Python}
    Python is a versatile and high-level programming language known for its simplicity and readability. It is widely used for a variety of applications, including web development, data analysis, artificial intelligence, scientific computing, and automation. It supports multiple programming paradigms, including procedural, object-oriented, and functional programming, and its large and active community contributes to a vast ecosystem of libraries and frameworks, making it a powerful tool for a wide range of tasks. It is one of the two languages supported by Ray, alongside Java.
    
    \subsection{Ray}
    To parallelize a function with Ray, import Ray and and initialize it with \textit{ray.init()}. Then decorate the function with \textit{@ray.remote} to declare that you want to run this function remotely. Lastly, call the function with \textit{.remote()} instead of calling it normally. This remote call yields a future, a Ray object reference, that you can then fetch with \textit{ray.get}:

    \begin{lstlisting}[language=Python]
import ray
ray.init()

@ray.remote
def f(x):
    return x * x

futures = [f.remote(i) for i in range(4)]
print(ray.get(futures)) # [0, 1, 4, 9]
    \end{lstlisting}

\subsection{Ray clusters}

\pagebreak
\subsection{The Matrix class}
Before implementing the parallelized algorithms for Ray, we write the code for the serial execution in the \textbf{Matrix} class (most of the functions defined in the class are auxiliary therefore were excluded for practical purposes):
\begin{lstlisting}[language=Python, label=Matrix class]
class Matrix:
    def __init__(self, data):
        if isinstance(data, list):
            self.data = [[round(val, 8) for val in row]
                                        for row in data]
        else:
            raise ValueError("Input must be a list")
    
    [...]

    def dot(A, B): [...]
    def det(self): [...]
    def rank(self): [...]
    def inv(self): [...]
\end{lstlisting}
Which we modify later to achieve parallelization with Ray.

\pagebreak
\subsubsection{Dot product}
\textbf{Matrix multiplication} is a binary operation that produces a matrix from two matrices. For matrix multiplication, the number of columns in the first matrix must be equal to the number of rows in the second matrix. The resulting matrix, known as the matrix \textbf{product}, has the number of rows of the first and the number of columns of the second matrix.
$$
    A = \begin{bmatrix}
        a_{11} & a_{12} & \dots  & a_{1n} \\
        a_{21} & a_{22} & \dots  & a_{2n} \\
        \vdots & \vdots & \ddots & \vdots \\
        a_{m1} & a_{m2} & \dots  & a_{mn}
    \end{bmatrix}, \quad
    B = \begin{bmatrix}
        b_{11} & b_{12} & \dots  & b_{1p} \\
        b_{21} & b_{22} & \dots  & b_{2p} \\
        \vdots & \vdots & \ddots & \vdots \\
        b_{n1} & b_{n2} & \dots  & b_{np}
    \end{bmatrix}
$$

The matrix multiplication of $A$ and $B$ is denoted as $C = A \cdot B$. The resulting matrix $C$ will have dimensions $m \times p$.

    The entry $c_{ij}$ of the resulting matrix $C$ is computed as follows:

    $$
        c_{ij} = a_{i1} \cdot b_{1j} + a_{i2} \cdot b_{2j} + \dots + a_{in} \cdot b_{nj} = \sum_{k=1}^{n} a_{ik} \cdot b_{kj}
    $$

    The resulting matrix $C$ can be expressed as:

    $$
        C = \begin{bmatrix}
            c_{11} & c_{12} & \dots  & c_{1p} \\
            c_{21} & c_{22} & \dots  & c_{2p} \\
            \vdots & \vdots & \ddots & \vdots \\
            c_{m1} & c_{m2} & \dots  & c_{mp}
        \end{bmatrix}
    $$

The definition above can be described as the following code:
\begin{lstlisting}[language=Python, caption={Dot function}]
@staticmethod
def dot(A, B):
    if A.is_matrix() and B.is_matrix():
        a_rows, a_cols = A.shape()
        b_rows, b_cols = B.shape()

        if a_cols == b_rows:
            result = [[0] * b_cols for _ in range(A.shape()[0])]
            for i in range(A.shape()[0]):
                for j in range(b_cols):
                    for k in range(a_cols):
                        result[i][j] += A.data[i][k] * B.data[k][j]
            return Matrix(result)
        else:
            raise ValueError(
                "Matrix dimensions do not match for dot product")
    else:
        raise ValueError("Dot product requires a Matrix object")
\end{lstlisting}
\vfill
To parallelize this function, we simply delegate the calculation made in line 12 to an auxiliary function \textit{dot\_calc} defined in the \textit{tasks.py} file:
\begin{lstlisting}[language=Python, caption={dot\_calc}]
@ray.remote
def dot_calc(A, B, i, j, k):
    return A.data[i][k] * B.data[k][j]
\end{lstlisting}

\begin{lstlisting}[language=Python, caption={Parallelized dot function}]
@staticmethod
def dot(A, B):
    if A.is_matrix() and B.is_matrix():
        a_rows, a_cols = A.shape()
        b_rows, b_cols = B.shape()

        if a_cols == b_rows:
            result = [[0] * b_cols for _ in range(a_rows)]
            futures = []

            for i in range(a_rows):
                for j in range(b_cols):
                    for k in range(a_cols):
                        # result[i][j] += A.data[i][k] * B.data[k][j]
                        futures.append(((i, j), t.dot_calc.remote(A, B, i, j, k)))
            
            for future in futures:
                i, j = future[0]
                result[i][j] += ray.get(future[1])

            return Matrix(result)
        else:
            raise ValueError(
                "Matrix dimensions do not match for dot product")
    else:
        raise ValueError("Dot product requires a Matrix object")
\end{lstlisting}

\subsubsection{Determinant}
\subsubsection{Rank}
\subsubsection{Inverse}
\pagebreak
\section{Result of tests}


\section{Conclusion}

\end{document}