\documentclass{beamer}
\usetheme{metropolis}

\AtBeginSection[]
{
  \begin{frame}
    \frametitle{Table of Contents}
    \tableofcontents[currentsection]
  \end{frame}
}

\title{Distributed computation of linear algebra operations}
\subtitle{Distributed systems and networks laboratory}
\author{
    Gabriele Aloisio [503264] \and
    Samuel Giacomo Raffa [matricola]
    }

\institute{Università degli studi di Messina}
\date{2023}
\logo{\includegraphics[height=1cm]{unime.png}}

\begin{document}
\maketitle

\section{Introduction}
\begin{frame}{Overview of distributed systems and networks}

\end{frame}


\begin{frame}{Importance of distributed computation in solving large-scale problems}

\end{frame}

\begin{frame}{Motivation for using Python and the Ray library}

\end{frame}


\section{Background on linear algebra operations}
\begin{frame}{Explanation of common linear algebra operations}

\end{frame}

\begin{frame}{Challenges in performing these operations on large datasets}

\end{frame}

\begin{frame}{}

\end{frame}

\section{The Ray library}
\begin{frame}{Overview of the Ray library and its capabilities}

\end{frame}

\begin{frame}{Key features and advantages of using Ray for distributed computation}

\end{frame}

\section{Implementation in Python}
\begin{frame}{List of operation}
    We will showcase the following operations:
    \begin{itemize}
        \item{Product}
        \item{Determinant}
        \item{Inverse}
        \item{Rank}
    \end{itemize}
\end{frame}

\subsection{The Matrix and RayMatrix classes}
\begin{frame}{The Matrix and RayMatrix classes}

\end{frame}

\begin{frame}{The Matrix class}

\end{frame}

\begin{frame}{The RayMatrix class}

\end{frame}

\subsection{Operations}

\end{document}