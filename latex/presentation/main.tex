\documentclass{beamer}
\usetheme{metropolis}
\usepackage{verbatim}

\AtBeginSection[]
{
  \begin{frame}
    \frametitle{Table of Contents}
    \tableofcontents[currentsection]
  \end{frame}
}

\title{Distributed computation of linear algebra operations}
\subtitle{Distributed systems and networks laboratory}
\author{
    Gabriele Aloisio [503264] \and
    Samuel Giacomo Raffa [matricola]
    }

\institute{Università degli studi di Messina}
\date{2023}
\logo{\includegraphics[height=1cm]{unime.png}}

\begin{document}
\maketitle

\section{Introduction}
\begin{frame}{Matrix operations}
    Matrix operations are fundamental in numerous scientific and computational domains, serving as the building blocks for various applications. However, traditional sequential computation on a single machine often becomes a bottleneck, limiting the speed and scalability of matrix calculations.
\end{frame}

\begin{frame}{Matrix operations}
    To address this problem, we are going to use an open-source distributed computing framework called \alert{Ray}.
\end{frame}

\begin{frame}{Overview of distributed systems and networks}
    A \alert{distributed system} consists of multiple interconnected computers that collaborate and coordinate their activities to achieve a common goal. These systems are designed to tackle  tasks that cannot be efficiently computed by a single machine.

    \alert{Networks} serve as the backbone of distributed systems, enabling communication among the connected nodes. Network infrastructures enable coordination, data sharing, and synchronization across distributed systems, regardless of their physical locations.
\end{frame}


\begin{frame}{Issues with distributed systems}
    Distributed systems also present unique challenges:
    \begin{itemize}
        \item{Data consistency}
        \item {Fault tolerance}
        \item {Load balancing}
        \item {Network latency}
    \end{itemize}
\end{frame}

\begin{frame}{Solving challenges with Ray}
    When it comes to matrix operation computation, using Ray for distributed computation offers several advantages over performing the operations on a single machine:
    \begin{itemize}
        \item Faster execution
        \item Scalability
        \item Fault tolerance
        \item Resource utilization
    \end{itemize}
\end{frame}

\begin{frame}{Faster execution}
    With Ray you can use multiple machines and CPUs to compute operations. This significantly reduces the computation time compared to a single machine. Each machine can work on a subset of the matrix data, completing the calculations in parallel. This distributed approach can lead to substantial speedups.
\end{frame}

\begin{frame}{Scalability}
    As the size of the matrices grows, a single machine may struggle to handle the computational demands due to memory limitations or processing power constraints. Ray allows you to scale horizontally by adding more machines to the distributed setup.
\end{frame}

\begin{frame}{Fault tolerance}
    Ray offers fault tolerance mechanisms that ensure the continuity of computation even in the presence of failures. If a machine participating in the distributed computation fails, Ray can automatically redistribute the workload to other available machines.
\end{frame}

\begin{frame}{Resource utilization}
    With Ray, each machine contributes its processing power and memory capacity to the overall computation. This efficient utilization of resources allows you to make the most of the available hardware infrastructure, compared to a single machine that may be underutilized.
\end{frame}

\section{Implementation in Python}
\begin{frame}{The Matrix and RayMatrix classes}
    Before implementing the parallelized algorithms for Ray, we wrote the code for the serial execution in the \alert{Matrix} class:
\end{frame}

\begin{frame}[fragile]{The Matrix class}
    \begin{verbatim}
        class Matrix:
            def __init__(self, elements):
                [...]
                self.elements = elements
            def inv(self): [...]
            def det(self): [...]
            def product(self, b): [...]
            def rank(self, order): [...]
            [...]

    \end{verbatim}

    (Most of the functions defined in the class are auxiliary therefore were excluded for practical purposes)
\end{frame}

\begin{frame}{The RayMatrix class}
    The \alert{RayMatrix} class is an extension of the Matrix class that inherits its attributes and methods. Operations that had to be parallelized were overrided and new auxiliary methods (methods that start with \textit{task\textunderscore}) decorated with \textbf{@ray.remote} were added:
\end{frame}

\begin{frame}[fragile]{The RayMatrix class}
    \begin{tiny}
        \begin{verbatim}
        class RayMatrix(Matrix):
            def __init__(self, elements):
                super().__init__(elements)
            
            @ray.remote
            def task_rank_det(submatrix, j):

            @ray.remote
            def task_get_square_submatrix(self, start_row, start_col, row, order):

            @ray.remote
            def task_det(self, elements, i):

            @ray.remote
            def task_inv_cof(self, a, i, j):

            @staticmethod
            @ray.remote
            def task_multiply(a, b, i, j, k):
            
            @staticmethod
            @ray.remote
            def task_sum(results):
        
    \end{verbatim}
    \end{tiny}
\end{frame}

\section{Operations}
\begin{frame}{List of operation}
    We will showcase the following operations:
    \begin{itemize}
        \item{Multiplication}
        \item{Determinant}
        \item{Inverse}
        \item{Rank}
    \end{itemize}
\end{frame}

\subsection{Multiplication}
\begin{frame}{Multiplication}
    \alert{Matrix multiplication} is a binary operation that produces a matrix from two matrices. For matrix multiplication, the number of columns in the first matrix must be equal to the number of rows in the second matrix. The resulting matrix, known as the matrix \textbf{product}, has the number of rows of the first and the number of columns of the second matrix.
    $$
        A = \begin{bmatrix}
            a_{11} & a_{12} & \dots  & a_{1n} \\
            a_{21} & a_{22} & \dots  & a_{2n} \\
            \vdots & \vdots & \ddots & \vdots \\
            a_{m1} & a_{m2} & \dots  & a_{mn}
        \end{bmatrix}, \quad
        B = \begin{bmatrix}
            b_{11} & b_{12} & \dots  & b_{1p} \\
            b_{21} & b_{22} & \dots  & b_{2p} \\
            \vdots & \vdots & \ddots & \vdots \\
            b_{n1} & b_{n2} & \dots  & b_{np}
        \end{bmatrix}
    $$
\end{frame}

\begin{frame}{Multiplication}
    The matrix multiplication of $A$ and $B$ is denoted as $C = A \cdot B$. The resulting matrix $C$ will have dimensions $m \times p$.

    The entry $c_{ij}$ of the resulting matrix $C$ is computed as follows:

    $$
        c_{ij} = a_{i1} \cdot b_{1j} + a_{i2} \cdot b_{2j} + \dots + a_{in} \cdot b_{nj} = \sum_{k=1}^{n} a_{ik} \cdot b_{kj}
    $$

    The resulting matrix $C$ can be expressed as:

    $$
        C = \begin{bmatrix}
            c_{11} & c_{12} & \dots  & c_{1p} \\
            c_{21} & c_{22} & \dots  & c_{2p} \\
            \vdots & \vdots & \ddots & \vdots \\
            c_{m1} & c_{m2} & \dots  & c_{mp}
        \end{bmatrix}
    $$
\end{frame}

\begin{frame}[fragile]{Multiplication with Ray (1)}
    \begin{small}
        \begin{verbatim}
@staticmethod
def product(a, b):
    a_columns = a.size()["columns"]
    a_rows = a.size()["rows"]

    b_columns = b.size()["columns"]

    if a_rows != b_columns:
        raise ValueError(
            """
            Number of columns of the first matrix 
            must match the number of rows of the second matrix
            """
            )
    else:
    ...
    
    \end{verbatim}
    \end{small}
\end{frame}

\begin{frame}[fragile]{Multiplication with Ray (2)}
    \begin{tiny}
        \begin{verbatim}
a_elements = a.get()  # Get the elements of matrix a
b_elements = b.get()  # Get the elements of matrix b

result = [[0] * b_columns for _ in range(a_rows)]  # Create a result matrix filled with zeros
tasks = []  # Initialize an empty list to store the tasks

# Iterate over rows of matrix a
for i in range(a_rows):
    # Iterate over columns of matrix b
    for j in range(b_columns):
        elements_to_multiply = []  # Initialize an empty list to store the elements to be multiplied
        
        # Iterate over columns of matrix a and rows of matrix b
        for k in range(a_columns):
            # Add a task to multiply the elements at position (i, k) and (k, j)
            elements_to_multiply.append(
                RayMatrix.task_multiply.remote(a=a_elements, b=b_elements, i=i, j=j, k=k)
            )
        
        # Add the task to compute the sum of multiplied elements
        tasks.append(RayMatrix.task_sum.remote(results=ray.get(elements_to_multiply)))
    \end{verbatim}
    \end{tiny}
\end{frame}

\begin{frame}[fragile]{Multiplication with Ray (3)}
    \begin{verbatim}
# Get the results of all the tasks
results = ray.get(tasks)

# Fill the result matrix with the computed results
for i in range(a_rows):
    for j in range(b_columns):
        result[i][j] = results.pop(0)

return RayMatrix(result)
            
    \end{verbatim}
\end{frame}

\begin{frame}[fragile]{Multiplication with Ray(4)}
    \begin{verbatim}
        @staticmethod
        @ray.remote
        def task_multiply(a, b, i, j, k):
            return a[i][k] * b[k][j]
    \end{verbatim}
\end{frame}

\subsection{Determinant}
\begin{frame}{Determinant}
    The \alert{determinant} is a scalar value that is a function of the entries of a square matrix. It characterizes some properties of the matrix and the linear map represented by the matrix. In particular, the determinant is nonzero if and only if the matrix is \textbf{invertible}.

\end{frame}

\begin{frame}{Determinant - Laplace method}
    To compute the determinant of matrices with order greater than 2 we are going to use the \alert{Laplace method}.


    Let $A$ be a square matrix of size $n \times n$. The Laplace expansion of the determinant along the $i$th row is given by:

    $$
        \det(A) = a_{i1} C_{i1} + a_{i2} C_{i2} + \ldots + a_{in} C_{in}
    $$

    where $C_{ij}$ denotes the cofactor of the element $a_{ij}$.
\end{frame}

\begin{frame}{Determinant - Cofactor}
    The \alert{cofactor} $C_{ij}$ is calculated as follows:

    $$
        C_{ij} = (-1)^{i+j} \cdot \det(M_{ij})
    $$

    where $\det(M_{ij})$ represents the determinant of the submatrix obtained by deleting the $i$th row and $j$th column from matrix $A$.
\end{frame}

\begin{frame}{Determinant}
    Using the Laplace method, the determinant can be calculated recursively by expanding along any row or column until a $2 \times 2$ matrix is reached, for which the determinant can be directly computed.

    The Laplace method provides an alternative approach for determining the determinant of a matrix and can be particularly useful for matrices of larger sizes.
\end{frame}

\begin{frame}[fragile]{Determinant with Ray (1)}
    \begin{tiny}
        \begin{verbatim}
def det(self):
if self.is_square():
    size = self.size()["rows"]
    a = self.get()

    if size == 1:
        return a[0][0]

    elif size == 2:
        return (a[0][0] * a[1][1]) - (a[0][1] * a[1][0])

    else:
        sum = 0

        for i in range(1, size):
            print(i)

            futures = self.task_det.remote(self=self, elements=a, i=i)
            print(ray.get(futures))

            sum += ray.get(futures)

        return sum

else:
    raise ValueError("Cannot compute determinant of a non-square matrix")
        
            \end{verbatim}
    \end{tiny}
\end{frame}

\begin{frame}[fragile]{Determinant with Ray (2)}

    \begin{small}
        \begin{verbatim}
@ray.remote
def task_det(self, elements, i):
    size = len(elements)
    submatrix_det_sum = 0
    mats = self.get_square_submatrices(i)

    for j in range(1, size):
        submatrix_det = mats[j].det2()
        submatrix_det_sum += 
            elements[i][j] * ((-1) ** (i + j + 2)) 
                * submatrix_det

    return submatrix_det_sum
    \end{verbatim}
    \end{small}

\end{frame}

\end{document}