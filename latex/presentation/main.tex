\documentclass{beamer}
\usetheme{metropolis}

\AtBeginSection[]
{
  \begin{frame}
    \frametitle{Table of Contents}
    \tableofcontents[currentsection]
  \end{frame}
}

\title{Distributed computation of linear algebra operations}
\subtitle{Distributed systems and networks laboratory}
\author{
    Gabriele Aloisio [503264] \and
    Samuel Giacomo Raffa [matricola]
    }

\institute{Università degli studi di Messina}
\date{2023}
\logo{\includegraphics[height=1cm]{unime.png}}

\begin{document}
\maketitle

\section{Introduction}
\begin{frame}{Matrix operations}
    Matrix operations are fundamental in numerous scientific and computational domains, serving as the building blocks for various applications. However, traditional sequential computation on a single machine often becomes a bottleneck, limiting the speed and scalability of matrix calculations.
\end{frame}

\begin{frame}{Matrix operations}
    To address this problem, we are going to use an open-source distributed computing framework called \alert{Ray}.
\end{frame}

\begin{frame}{Overview of distributed systems and networks}
A \alert{distributed system} consists of multiple interconnected computers that collaborate and coordinate their activities to achieve a common goal. These systems are designed to tackle  tasks that cannot be efficiently computed by a single machine.

\alert{Networks} serve as the backbone of distributed systems, enabling communication among the connected nodes. Network infrastructures enable coordination, data sharing, and synchronization across distributed systems, regardless of their physical locations.
\end{frame}


\begin{frame}{Issues with distributed systems}
    Distributed systems also present unique challenges:
    \begin{itemize}
        \item{Data consistency}
        \item {Fault tolerance}
        \item {Load balancing}
        \item {Network latency}
    \end{itemize}
\end{frame}

\begin{frame}{Solving challenges with Ray}
    When it comes to matrix operation computation, using Ray for distributed computation offers several advantages over performing the operations on a single machine:
    \begin{itemize}
    \item Faster execution
    \item Scalability
    \item Fault tolerance
    \item Resource utilization
    \end{itemize}
\end{frame}

\begin{frame}{Faster execution}
    With Ray you can use multiple machines and CPUs to compute operations. This significantly reduces the computation time compared to a single machine. Each machine can work on a subset of the matrix data, completing the calculations in parallel. This distributed approach can lead to substantial speedups.
\end{frame}

\begin{frame}{Scalability}
    As the size of the matrices grows, a single machine may struggle to handle the computational demands due to memory limitations or processing power constraints. Ray allows you to scale horizontally by adding more machines to the distributed setup.
\end{frame}

\begin{frame}{Fault tolerance}
    Ray offers fault tolerance mechanisms that ensure the continuity of computation even in the presence of failures. If a machine participating in the distributed computation fails, Ray can automatically redistribute the workload to other available machines.
\end{frame}

\begin{frame}{Resource utilization}
    With Ray, each machine contributes its processing power and memory capacity to the overall computation. This efficient utilization of resources allows you to make the most of the available hardware infrastructure, compared to a single machine that may be underutilized.
\end{frame}


\section{Background on linear algebra operations}
\begin{frame}{Explanation of common linear algebra operations}

\end{frame}

\begin{frame}{Challenges in performing these operations on large datasets}

\end{frame}

\begin{frame}{}

\end{frame}

\section{The Ray library}
\begin{frame}{Overview of the Ray library and its capabilities}

\end{frame}

\begin{frame}{Key features and advantages of using Ray for distributed computation}

\end{frame}

\section{Implementation in Python}
\begin{frame}{List of operation}
    We will showcase the following operations:
    \begin{itemize}
        \item{Product}
        \item{Determinant}
        \item{Inverse}
        \item{Rank}
    \end{itemize}
\end{frame}

\subsection{The Matrix and RayMatrix classes}
\begin{frame}{The Matrix and RayMatrix classes}

\end{frame}

\begin{frame}{The Matrix class}

\end{frame}

\begin{frame}{The RayMatrix class}

\end{frame}

\subsection{Operations}

\end{document}